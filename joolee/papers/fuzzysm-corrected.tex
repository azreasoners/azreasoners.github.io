% intuition ovf semantics? 

% 13 pages + bib

\documentclass[runningheads]{llncs}
\usepackage{times}
\usepackage{amsmath}
\usepackage{amssymb}
\usepackage{color}
%\usepackage[compact]{\titlesec}
%\usepackage[small,bf,belowskip=-10pt,aboveskip=0pt]{caption}

\setlength{\intextsep}{2mm}

\newcommand{\cblu}{\color{blue}}
\newcommand{\cred}{\color{red}}
\newcommand{\cmag}{\color{magenta}}
\newcommand{\cgre}{\color{green}}
\newcommand{\cbla}{\color{black}}

\long\def\BOC#1\EOC{\message{(Commented text )}}
\long\def\BOCC#1\EOCC{\message{(Commented text )}}
\long\def\BOCCC#1\EOCCC{\message{(Commented text )}}
\long\def\optional#1{\empty}
%\long\def\optional#1{[#1]}
%\long\def\NB#1{[{\cblu {\bf N.B.} #1}]}
\long\def\NB#1{}


\def\o{\overline}
\def\ar{\leftarrow}
\def\bi{\begin{itemize}}
\def\ei{\end{itemize}}
\def\beq{\begin{equation}}
\def\eeq#1{\label{#1}\end{equation}}
\def\ba{\begin{array}}
\def\ea{\end{array}}
\def\i#1{\hbox{\it #1\/}}
\def\mi#1{\mathit{#1}}
%\def\mi#1{\hbox{\mathit #1\/}}
\def\is#1{\hbox{\footnotesize {\it #1\/}}}
\def\iss#1{{\hbox{\tiny {\it #1\/}}}}
\def\sm{\hbox{\rm SM}}
\def\C{{\cal C}}
\def\no{\i{not}}
\def\sneg{\sim\!\!}
\def\implies{\supset}
\def\ar{\leftarrow}
\def\rar{\rightarrow}
\def\lrar{\leftrightarrow}
\def\no{\i{not}}
\def\mvis{\!=\!}
\def\false{\hbox{\sc false}}
\def\true{\hbox{\sc true}}
\def\nec{\mbox{\large\boldmath $\;\Leftarrow\;$}}
\def\i#1{\hbox{\itshape #1\/}}
\def\r#1#2{\frac{\textstyle #1}{\textstyle #2}}
\def\proof{\noindent{\bf Proof}.\hspace{3mm}}
\def\qed{\quad \vrule height7.5pt width4.17pt depth0pt \medskip}
     \def\smmodels{\models_{\text{\sm}}}
\def\Lrar{\Leftrightarrow}
\DeclareSymbolFont{AMSa}{U}{msa}{m}{n}
\DeclareMathSymbol{\square}{\mathord}{AMSa}{"03}
\def\wh{\widehat}
\def\mvis{\!=\!}
\def\mu#1{\mathit{\underline{#1}}}

\def\fand{\otimes}
\def\for{\oplus}
\def\fneg{\neg}
\def\frar{\rar}

\newtheorem{con}{Conjecture}
\newtheorem{prop}{Proposition}
\newtheorem{thm}{Theorem}
\newtheorem{cor}{Corollary}

\linespread{0.97}
\addtolength{\textwidth}{2mm}

\title{Stable Models of Fuzzy Propositional Formulas}

\author{Joohyung Lee and Yi Wang}
\institute{
School of Computing, Informatics, and Decision Systems Engineering \\
Arizona State University, Tempe, USA \\
{\tt \{joolee,ywang485\}@asu.edu}}


\begin{document}

\maketitle


%---------------------------------------------------------------------

\begin{abstract}
We introduce the stable model semantics for fuzzy propositional
formulas, which generalizes both fuzzy propositional logic and the
stable model semantics of Boolean propositional formulas. 
Combining the advantages of both formalisms, the introduced
language allows highly configurable default reasoning involving fuzzy
truth values. We show that several properties of Boolean stable models
are naturally extended to this formalism, and discuss how it is related to other approaches to combining fuzzy logic and the stable model semantics. 
\end{abstract}

%---------------------------------------------------------------------
\section{Introduction} \label{sec:intro}\optional{sec:intro}
%---------------------------------------------------------------------

Answer set programming (ASP) \cite{lif08} is a widely applied
declarative programming paradigm for the design and implementation of
knowledge intensive applications. One of the attractive features of
ASP is its capability to model the nonmonotonic aspect of
knowledge. However, as its mathematical basis, the stable model
semantics, is restricted to Boolean values, it is too rigid to
represent imprecise and vague information. 
Fuzzy logic, as a form of many-valued logic,
can handle vague information by interpreting propositions with a truth
degree in the interval of real numbers $[0,1]$.
%; the higher the degree assigned to a proposition, the more true it
%is. 
The availability of various fuzzy operators gives the user great
flexibility in combining truth degrees. However, the
semantics of fuzzy logic is monotonic and is not flexible enough to
handle default reasoning as allowed in answer set programming.

Both the stable model semantics and fuzzy logic are generalizations of
classical propositional logic in different ways. While they do not
subsume each other, it is clear that many real-world problems require
both their strengths. This led to the body of work on combining
fuzzy logic and the stable model semantics, known as fuzzy answer set
programming (e.g., \cite{lukasiewicz06fuzzy,janssen12reducing,vojtas01fuzzy,damasio01monotonic,medina01multi-adjoint,damasio01antitonic,nieuwenborgh07anintroduction,madrid08towards}).
However, most work considers simple rule forms and do not allow 
connectives nested arbitrarily as in fuzzy logic. 

Unlike existing work on fuzzy answer set semantics, in this paper, we
extend the general stable model semantics from~\cite{ferraris11stable}
to many-valued propositional formulas. The syntax of this language is
the same as the syntax of fuzzy propositional logic. The semantics, on
the other hand, distinguishes {\sl stable} models from non-stable models.
The language is a proper generalization of both fuzzy propositional logic and Boolean propositional formulas under the stable model semantics.
This generalization is not simply a pure theoretical pursuit, but has
practical use in conveniently modeling defaults involving fuzzy truth
values in dynamic domains. For example, consider modeling dynamics of
{\em trust} in social
network. People trust each other in different degrees under some
normal assumptions. If person $A$ trusts person $B$, then $A$
tends to trust person $C$ whom $B$ trusts to a degree which is
positively correlated to the degree to which $A$ trusts $B$ and the
degree to which $B$ trusts $C$. By default, the trust degrees
would not change, but may decrease when a conflict arises between people.
%if a conflict between $A$ and $B$ happens, there will be
%less trust between A and B, depending on how severe the fight is.
Modeling such a domain requires expressing defaults involving fuzzy
truth values. We demonstrate that such examples can be conveniently
modelled in our proposed language by taking advantage of its
generality over the existing approaches to fuzzy ASP.

%We also show that existing proposals of the fuzzy ASP can be embedded
%into our general language. Further, the well known mathematical
%results in traditional answe set programming 

The paper is organized as follows. Section~\ref{sec:prelim} reviews
the syntax and the semantics of fuzzy propositional logic
we discuss in the paper, as well as the stable model semantics of
classical propositional formulas. Section~\ref{sec:definition}
presents the stable model semantics of fuzzy propositional formulas
along with examples, including the above
%followed by Section~\ref{sec:further} that
%formalizes 
trust example in the proposed language. 
Section~\ref{sec:rel-bool}  relates our fuzzy stable model semantics
to the Boolean stable model semantics, and
Section~\ref{sec:rel-fuzzyasp} relates it to other approaches to fuzzy ASP. Section~\ref{sec:properties} shows that several well-known properties
of the Boolean stable model semantics can be easily extended to our fuzzy
stable model semantics. Section~\ref{sec:related_work} discusses other
related work.

%A longer version with the complete proofs is available from
%\url{http://reasoning.eas.asu.edu/papers/fuzzysm-long.pdf}.

\vspace{-0.3cm}
%---------------------------------------------------------------------
\section{Preliminaries} \label{sec:prelim} \optional{sec:prelim}
%---------------------------------------------------------------------

%---------------------------------------------------------------------
\subsection{Review: Stable Models of Classical Propositional
  Formulas}  \label{ssec:review-sm} \optional{ssec:review-sm}
%---------------------------------------------------------------------

%The definition of a stable model given in~\cite{ferraris11stable} is
%in terms of non-existence of an witness for the modified formula. 
%Consider any classical propositional formula. 
We review the definition of a stable model
from~\cite{ferraris11stable} by limiting attention to the syntax of
propositional formulas. Instead of defining stable models in terms of second-order logic as in~\cite{ferraris11stable} , we express the same concept using auxiliary atoms that do not belong to the original signature. This slight reformulation
will simplify our efforts in extending the stable model semantics to fuzzy
propositional formulas without resorting to ``second-order fuzzy
logic.''

Let $\sigma$ be a classical propositional signature, 
let ${\bf p}=(p_1,\dots, p_n)$ be a list of distinct atoms belonging
to~$\sigma$, and let ${\bf q}=(q_1,\dots,q_n)$ be a list of new, distinct
propositional atoms not belonging to~$\sigma$.  
%By $\sigma\cup{\bf q}$ we denote the signature obtained from~$\sigma$ by
%adding all members of ${\bf q}$. 
%
For two interpretations $I$ and $J$ of $\sigma$ that agree on all
atoms in $\sigma\setminus {\bf p}$, $I\cup J^{\bf p}_{\bf q}$ denotes the
interpretation of $\sigma\cup{\bf q}$ that 
\bi
\item  agrees with $I$ on all atoms in $\sigma$, and
\item  for each atom $q_i\in {\bf q}$, 
       $(I\cup J^{\bf p}_{\bf q})(q_i) = J(p_i)$.\footnote{%
$I(p)$ denotes the truth value of $p$ under $I$.
We identify a list with a set if there is no confusion.} 
\ei


% \bi
% \item  for each atom $p\in\sigma$, 
%        $(I\cup J^{\bf p}_{\bf q})(p) = I(p)$; 
% \item  for each atom $q\in {\bf q}$, 
%        $(I\cup J^{\bf p}_{\bf q})(q) = J(p)$.\footnote{%
% $I(p)$ denotes the truth value of $p$ under $I$.
% We identify a list with a set if there is no confusion.} 
% \ei

For any classical propositional formula $F$ of signature $\sigma$,
$F^*({\bf q})$ is a classical propositional formula of signature
$\sigma\cup{\bf q}$ that is defined recursively as follows: 
\begin{itemize}
\item  $p_i^* = q_i$ for each $p_i\in{\bf p}$; 
\item  $F^* = F$ for any atom $F\not\in {\bf p}$; 
\item  $\bot^* = \bot$;\ \ \ \  $\top^* = \top$; 
\item  $(\neg F)^* = \neg F$;
\item  $(F\land G)^* = F^* \land G^*$; \qquad
       $(F\lor G)^* = F^* \lor G^*$;
\item  $(F\rar G)^* = (F^* \rar G^*)\land (F \rar G)$.
\end{itemize}

Let $I$ and $J$ be two interpretations of~$\sigma$, and let ${\bf p}$
be a subset of $\sigma$. We say $J\le^{\bf p} I$ if 
\begin{itemize}
\item  $J$ and $I$ agree on all atoms not in ${\bf p}$, and
\item  for all $p\in {\bf p}$, if $J\models p$, then $I\models p$.
\end{itemize}
We say $J<^{\bf p} I$ if $J\le^{\bf p} I$ and $J\ne I$.


%For a formula $F$, suppose $\sigma$ is the underlying signature, which
%contains atoms $p_1, ..., p_n$ that appear in $F$. By $\sigma^*$ we
%denote the signature obtained from $\sigma$ by replacing each atom
%$p_i$ with $p_i^*$.
%For an interpretation $I$ of $\sigma$, by $I^*$ we denote an
%interpretation of $\sigma^*$ obtained from $\sigma$ by replacing each
%atom $p_i$ by $p_i^*$. 


\begin{definition}\label{def:cl-sm}\optional{def:cl-sm}
An interpretation $I$ is a {\em stable model} of $F$ relative to
${\bf p}$ (denoted $I\models\sm[F; {\bf p}]$) 
\bi
\item if $I\models F$, and
\item there is no interpretation $J$ such that $J<^{\bf p} I$ and $I\cup J^{\bf p}_{\bf q}\models F^*({\bf q})$.
\ei
\end{definition}


\begin{example}
Consider a logic program 
\[
\ba l
  p \ar \no\ q, \ \ \ 
  q \ar \no\ p 
\ea 
\]
which is understood as an alternative notation for propositional
formula $F_1 = (\neg q\rar p)\land (\neg p\rar q)$.
$F_1^*(u,v)$ is 
\hbox{$(\neg q\rar u)\land (\neg q\rar p)\land (\neg p\rar
   v)\land (\neg p\rar q)$}.
% which is equivalent to 
% \[ 
%    (\neg q\rar u)\land (\neg q\rar p)\land (\neg p\rar
%    v)\land (\neg p\rar q)
% \]
% under the assumption $F_1$. 
%
We check that $I_1 = \{p\}$ (that is, $p$ is $\true$ and $q$ is
$\false$) \footnote{%
We identify a propositional interpretation with the set of atoms that
are true in it. }
is a stable model of $F_1$ (relative to
$\{p,q\}$):  $I_1$ satisfies $F_1$, and $\emptyset$ is the
only interpretation $J$ such that $J<^{pq} I_1$. However, 
$I_1\cup J^{pq}_{uv} = \{p\}$ does not satisfy $F_1^*(u,v)$
because it does not satisfy the first conjunctive term of $F_1^*(u,v)$.
%
Similarly, we can check that $\{q\}$ is another stable model of $F_1$.
\end{example} 

%---------------------------------------------------------------------
\subsection{Review: Fuzzy Logic}  \label{ssec:review-fuzzy}
  \optional{ssec:review-fuzzy}
%---------------------------------------------------------------------

%The syntax of fuzzy propositional logic is an extension of the syntax
%of classical propositional logic. 
%Logical operations of conjunction, disjunction, implication and
%negation are generalized in fuzzy logic. 

Let $\sigma$ be a fuzzy propositional signature, which is a set of
symbols called {\em fuzzy atoms}. In addition, we assume the presence of a
set $\mathbb{C}$ of fuzzy conjunction symbols, a set $\mathbb{D}$ of
fuzzy disjunction symbols, a set $\mathbb{N}$ of fuzzy negation
symbols, and a set $\mathbb{I}$ of fuzzy implication symbols. 
%We also assume the presence of numeric constants $\bar{c}$ for each real
%number $c$ in $[0,1]$.

%We assume a set of symbols for
%``t-norms'', denoted $\otimes$, a se of conjunction symbols
%$T$, disjunction symbols $C$, implication symbols $I$ and negation
%symbols $N$. 
A {\em fuzzy (propositional) formula} of $\sigma$ is defined recursively as
follows.
%
\begin{itemize}
\item every fuzzy atom $p\in\sigma$ is a fuzzy formula;

\item every numeric constant $\overline{c}$ where $c$ is a real number in
  $[0, 1]$ is a fuzzy formula;

\item if $F$ is a fuzzy formula, then $\fneg F$ is a fuzzy formula, where $\fneg \in
  \mathbb{N}$;

\item if $F$ and $G$ are fuzzy formulas, then $F\fand G$, $F\for G$
  and $F\frar G$ are fuzzy formulas, where $\fand\in\mathbb{C}$,
  $\for\in\mathbb{D}$, and $\frar\ \in\mathbb{I}$.
\end{itemize}

The models of a fuzzy formula are defined as follows
\cite{hajek98mathematics}.
%
The {\em fuzzy truth values} are the real numbers in the range $\left[0,
  1\right]$. 
A \emph{fuzzy interpretation} $I$ of~$\sigma$ is a mapping
from $\sigma$ into $\left[0, 1\right]$.

The fuzzy operators are functions mapping one or a pair of truth values into
a truth value. Among the operators, $\fneg$ denotes a function from
$\left[0, 1\right]$ into $\left[0, 1\right]$; 
$\fand$, $\for$, and $\frar$ denote functions from 
$\left[0, 1\right]\times\left[0, 1\right]$ into $\left[0,
  1\right]$.
%
The actual mapping performed by each operator can be defined in many
different ways, but all of them satisfy the following conditions, 
which imply that the operators are generalizations of the
corresponding classical propositional connectives:\footnote{
We say that a function $f$ of arity $n$ is \emph{increasing in
  its $i$-th argument} ($1\leq i\leq n$) if
$f(arg_1,\dots, arg_i,\dots, arg_n)\leq
f(arg_1,\dots,arg_i^\prime,\dots, arg_n)$ for all arguments such that
$arg_i \leq arg_i^\prime$; 
$f$ is said to be \emph{increasing} if it is increasing in all its
arguments. The definition of \emph{decreasing} is similar. 
% We say that a function $f(arg_1,\dots,arg_n)$ is \emph{increasing in
%   its $i$-th argument} ($1\leq i\leq n$) if
% $f(arg_1,\dots, arg_i,\dots, arg_n)\leq
% f(arg_1,\dots,arg_i^\prime,\dots, arg_n)$ for all arguments 
% $arg_i \leq arg_i^\prime$; 
% $f$ is said to be \emph{increasing} if it is increasing in all its
% arguments. The definition of \emph{decreasing} is similar. 
}
\begin{itemize}
\item a fuzzy negation $\fneg$ is decreasing, and satisfies $\neg(0) =
  1$ and $\neg(1) = 0$;

\item a fuzzy conjunction $\fand$ is increasing, commutative,
  associative, and $\fand(1, x)=x$ for all $x \in \left[0, 1\right]$;
%  $\fand(1, x)=x$, and for all $x, y \in \left[0, 1\right]$,
%  $\fand(x, y) \geq x$;

\item a fuzzy disjunction $\for$ is increasing, commutative,
  associative, and $\for(0, x)=x$ for all $x\in \left[0, 1\right]$;

\item a fuzzy implication $\frar$ is decreasing in its first
  argument and increasing in its second argument; and $\frar(1, x) =
  x$ and $\frar(0, 0) = 1$ for all $x\in\left[0, 1\right]$.
\end{itemize}

Figure~\ref{fig:operators} lists some specific fuzzy operators that we
use in this paper.

%\vspace{-5mm}
\begin{figure}
\begin{center}
\begin{tabular}{| c | l | l |}
			\hline
			\textbf{Symbol} & \textbf{Name} & \textbf{Definition} \\
			\hline
			$\fand_l$ & \L ukasiewicz t-norm & $\fand_l(x, y)=max\left(x + y -1, 0\right)$ \\
			$\for_l$ & \L ukasiewicz t-conorm & $\for_l(x, y)=min\left(x + y, 1\right)$ \\
			\hline
			$\fand_m$ & minimum t-norm & $\fand_m(x, y)=min\left(x, y\right)$ \\
			$\for_m$ & maximum t-conorm & $\for_m(x, y)=max\left(x, y\right)$ \\
			\hline
			$\fand_p$ & product t-norm & $\fand_p(x, y)=x \cdot y$ \\
			$\for_p$ & product t-conorm & $\for_p(x, y)=x+ y -x\cdot y$ \\
			\hline
			$\neg_{\!s}$ & standard negator & $\neg_{\!s}(x) =
                        1-x$\\ \hline

			$\rar_r$ & the residual implicator of $\fand_m$ &
                        $\rar_r(x, y)=\begin{cases}1 & \text{if}\ x \leq y\\y
                          & \text{otherwise}\end{cases}$\\
			$\rar_s$ & the S-implicator induced by
                        $\neg_{\!s}$ and $\for_m$ &  $\rar_s(x, y) =
                        max\left(1-x, y\right)$\\  \hline
\end{tabular}
\caption{Some t-norms, t-conorms, negator, and implicators}
\label{fig:operators}
\end{center}
\end{figure}

%\vspace{-8mm}
The \emph{truth value} of a fuzzy formula $F$ under $I$, denoted $F^I$, is
defined recursively as follows:  \\[-5mm]
\begin{itemize}
\item for any atom $p \in \sigma$,\ \ $p^I = I(p)$;
\item for any numeric constant $\overline{c}$, $\o{c}^I = c$;
\item $(\neg F)^I = \neg(F^I)$;
\item $(F\fand G)^I = \fand(F^I, G^I)$;\ \ 
      $(F\for G)^I = \for(F^I, G^I)$;\ \  
      $(F\rar G)^I =\ \rar\!(F^I, G^I)$. 
\end{itemize}
(For simplicity, we identify the symbols for the fuzzy operators with
the truth value functions represented by them.) 

\begin{definition}\label{def:fuzzy-m}\optional{def:fuzzy-m}
We say that a fuzzy interpretation $I$ {\em satisfies} a fuzzy
formula~$F$ w.r.t. a threshold $y\in [0,1]$ if $F^I\ge y$, and
denote it by $I\models_y F$. We call $I$ a {\em fuzzy $y$-model} of~$F$. 
\end{definition}

We often omit the threshold $y$ when it is $1$.

%Two formulas $F$ and $G$ are said to be \emph{equivalent} if 
%$F^I= G^I$ for all interpretations~$I$.




%[[ residual ... ]]

\vspace{-0.3cm}

%---------------------------------------------------------------------
\section{Definition and Examples} \label{sec:definition} 
   \optional{sec:definition}
%---------------------------------------------------------------------

%We first extend the definition of a formula by introducing a reserved
%conjunction symbol $\fand_m$, associated with the function
%$min(x,y)$. 
%If $F$ and $G$ are formulas, $F \diamond G$
%is also a formula, and its truth value under an interpretation
%$I$ is defined by $(F \diamond G)^I=min\left(F^I, G^I\right)$.

We extend the notion of $J<^{\bf p} I$ in Section~\ref{ssec:review-sm} as
follows. 
For any two fuzzy interpretations $J$ and $I$ of the same
signature $\sigma$ and any subset ${\bf p}$ of $\sigma$, we say $J\le ^{\bf p} I$ if
\bi
\item  $J$ and $I$ agree on all fuzzy atoms not in ${\bf p}$, and
\item for all $p \in {\bf p}$, $p^J\le p^I$.
\ei
We say $J<^{\bf p} I$ if $J\le ^{\bf p} I$ and $J\ne I$.

As before, we assume a list ${\bf q}=(q_1,\dots,q_n)$ of new, distinct
fuzzy atoms that corresponds to ${\bf p}=(p_1,\dots,p_n)$, and
% ats$\sigma^*$ is the signature obtained from $\sigma$ by
%adding a list of new fuzzy atoms ${\bf q}=(q_1,\dots,q_n)$
%corresponding to ${\bf p}=(p_1,\dots,p_n)$ in $\sigma$. 
%$\sigma^*$ is the signature obtained from $\sigma$ by adding all
%members of ${\bf q}$. and 
define $I\cup J^{\bf p}_{\bf q}$ in the same way. That is, 
when $I$ and $J$ agree on all atoms in $\sigma\setminus {\bf p}$, 
$I\cup J^{\bf p}_{\bf q}$ denotes the interpretation of
$\sigma\cup{\bf q}$ that
\bi
\item  agrees with $I$ on all atoms in $\sigma$, and 
\item  for each $q_i\in {\bf q}$, 
       $(I\cup J^{\bf p}_{\bf q})(q_i)=J(p_i)$.
\ei

The definition of $F^*$ is also extended in a straightforward way: 
For any fuzzy formula~$F$ of signature~$\sigma$, $F^*({\bf q})$ is
defined as follows. 

                                                  
\begin{itemize}
\item  $p_i^* = q_i$ for each $p_i\in{\bf p}$; 
\item  $F^* = F$ for any atom $F\not\in {\bf p}$;
\item  $\o{c}^* = \o{c}$ for any numeric constant $\o{c}$;
\item  $(\neg F)^* = \neg F$;
\item  $(F\fand G)^* = F^*\fand G^*$; \qquad
       $(F\for G)^* = F^* \for G^*$;
\item  $(F\frar G)^* = (F^*\frar G^*)\fand_m (F \rar G)$. \footnote{%
Note the use of $\fand_m$ here; the value of
``conjunction'' of $(F^*\rar G^*)$ and $(F\rar G)$ needs not be smaller
than the value of $(F^*\rar G^*)$ and the value of $(F\rar G)$. It turns out that $\fand_m$ is the only t-norm that satisfies this property.}
\end{itemize}


\begin{definition}\label{def:fuzzy-sm}
A fuzzy interpretation $I$ is a fuzzy $y$-stable model of $F$ relative
to ${\bf p}$ (denoted $I\models_y\sm[F; {\bf p}]$) if 
\bi 
\item $I \models_y F$, and
\item there is no fuzzy interpretation $J$ such that  $J<^{\bf p} I$ and $I\cup
J^{\bf p}_{\bf q}\models_y F^*({\bf q})$.
\ei
\end{definition}

We often omit the threshold $y$ when it is $1$, and omit ${\bf p}$
if it contains all atoms in~$\sigma$. 

Clearly, when ${\bf p}$ is empty, Definition~\ref{def:fuzzy-sm}
reduces to the definition of a fuzzy model in
Definition~\ref{def:fuzzy-m} because there is no $J$ such
that $J<^\emptyset I$.

Also, Definition~\ref{def:fuzzy-sm} is very similar to the definition
of a stable model for classical propositional formulas in
Definition~\ref{def:cl-sm}. The main difference
is that simply in the latter, atoms may have various degrees of truth, and
accordingly the notion of $J<^{\bf p} I$ is more general. The
precise relationship between the definitions is discussed in
Section~\ref{sec:rel-bool}.


\begin{example}\label{ex:not_p_implies_q}\optional{ex:not_p_implies_q}
Consider the fuzzy formula $F = \neg_{\!s} p \rar_r q$ and the interpretation 
$I = \left\{(p, 0), (q, 0.6)\right\}$.  
$F^*(u,v)$ is  
\[
\ba {rl}
   ((\neg_{\!s} p)^* \rar_{r} q^*)\fand_m(\neg_{\!s}\ p \rar_r q)\ 
   =\ (\neg_{\!s} p \rar_r v)\fand_m(\neg_{\!s} p \rar_r q). 
\ea
\]

$I\models_{0.6} \sm[F;\ p,q]$. First, it is easy to see that 
$I\models_{0.6} F$, as 
\[  F^I = \rar_r((\neg_{\!s} p)^I, q^I) = \rar_r(1 -p^I, q^I) = \rar_r(1,
0.6) = 0.6.
\]

Suppose there exists $J<^{pq} I$ such that $I \cup J_{uv}^{pq}
\models_{0.6} F$, i.e.,
\[ 
\ba {rl} 
   F^*(u,v)^{I\cup J_{uv}^{pq}} 
   &= min\left(\rar_r(\neg_{\!s}(p^I), v^{I \cup J_{uv}^{pq}}),
               \rar_r(\neg_{\!s}(p^I), q^I)\right) \\
   &= min\left(\rar_r(1, q^J), 0.6\right) \\
   &= min\left(q^J, 0.6\right) \geq 0.6. 
\ea
\]
% \[ 
% \ba {rl} 
%    F^*(u,v)^{I\cup J_{uv}^{pq}} &= min\left(\rar_r(\neg_{\!s}(p^I),
%      v^{I \cup J_{uv}^{pq}}), \rar_r(\neg_{\!s}(p^I), q^I)\right) \\
%    &= min\left(\rar_r(1, v^{I \cup J_{uv}^{pq}}), 0.6\right) \\
%    &= min\left(v^{I \cup J_{uv}^{pq}}, 0.6\right) \geq 0.6. 
% \ea
% \]
So $q^J\geq 0.6$. 
%Since $q^J\leq q^I = 0.6$ follows from the
%assumption $J<^{pq} I$,
%we conclude that $q^J = 0.6$. However, 
This contradicts the assumption that $J <^{pq} I$. Therefore, such $J$
does not exist, and $I$ is a 0.6-stable model of $F$.
\end{example}


\begin{example}
$p$ and $\neg_{\!s}\neg_{\!s} p$ have the same fuzzy models, but their
stable models are different. This is similar to the fact that $p$ and
$\neg\neg p$ have different stable models according to the semantics
from~\cite{ferraris11stable}.

Clearly, any interpretation $I = \left\{(p, y)\right\}$, where $y$ is any positive real number in $\left[0, 1\right]$, is a $y$-stable model of $p$
relative to $\{p\}$. On the other hand, $I = \left\{(p, y)\right\}$ is not a $y$-stable model of $F = \neg_{\!s}\neg_{\!s} p$ relative to $\{p\}$.  
Formula $F^*(u)$ is $\neg_{\!s}\neg_{\!s} F$, and although $I\models_y F$,
%\[ 
%   F^*(q)^{I\cup J^{p}_{q}} = \neg_{\!s}(\neg_{\!s}(p^I)) = 
%   1-(1-p^I) = y
%\]
we have $I\cup J^p_u\models_y F^*(u)$ regardless of any  $J$. %such that $p^J<p^I$. 
%So $I$ is not a $y$-stable model of $\neg_{\!s}\neg_{\!s} p$,
%while it is a $y$-stable model of $p$.
% Although $I\models_y F$, 
% For $F = \neg_{\!s} \neg_{\!s} p$, $F^*(q)$ is $F$. 
% Any interpretation $I = \left\{(p, y)\right\}$, where $y$ is any positive
% real number in $\left[0, 1\right]$, is not a $y$-stable model of $F$
% relative to ${\bf p}=\{p\}$. Although $I\models_y F$, 
% %\[ 
% %   F^*(q)^{I\cup J^{p}_{q}} = \neg_{\!s}(\neg_{\!s}(p^I)) = 
% %   1-(1-p^I) = y
% %\]
% we have $I\cup J^p_q\models_y F^*(q)$ for any $J$ such that
% $p^J<p^I$. So $I$ is not a $y$-stable model of $\neg_{\!s}\neg_{\!s} p$,
% while it is a $y$-stable model of $p$.
% 
\end{example}


\begin{example}
Let $F_1 = p \rar_s p$ and $F_2 = \neg_{\!s} p \for_m p$. Their fuzzy
models are the same, but their stable models are not.
% are
%equivalent under the semantics of fuzzy logic, but not under the
%fuzzy stable model semantics. 
This is similar to the relation between $p\rar p$ and
$\neg p\lor p$ in the Boolean stable model semantics. 
%
Indeed, observe that \hbox{$F_1^*(u) = (p \rar_s p) \fand_m (u \rar_s u)$} 
and $F_2^*(u) = \neg_{\!s} p \for_m u$. 

The interpretation $I=\{(p, 1)\}$ is not a $1$-stable model
of $F_1$ relative to $p$, as witnessed by $J =\{(p,0)\}$. 
However, $I$ is a $1$-stable model of $F_2$ relative to~$p$: for
any~$J$,
\[
\ba {rl}
   F_2^*(u)^{I\cup J^{p}_{u}} = max\left(1-p^I, p^J\right) 
    = max\left(0, p^J\right) 
    = p^J.
\ea
\]
So, for $I\cup J^{p}_{u}$ to satisfy $F_2^*(u)$ to degree $1$, 
$p^J$ should be $1$.
Consequently, it is not possible to have $J <^{p} I$. 
\end{example}

The following example illustrates how the commonsense law of inertia involving fuzzy truth values can be represented. 

%representation of  useful construct that can be used
%to express the default reasoning. 

\begin{example}\label{ex:default}
Let $\sigma$ be $\{p, np, q, nq\}$ 
%\footnote{%
%Note that $\sneg\ $ is not a connective; it is just a part of the symbol 
%representing an atom.} 
and let $F$ be $F_1\fand_m F_2$, where $F_1$ represents that
$p$ and $np$ are complementary, i.e., the sum of their truth
values is~$1$:
\[ 
   F_1=\neg_{\!s}(p \fand_l np) \fand_m \neg_{\!s} \neg_{\!s}(p \for_l np).  
\]
$F_2$ represents that by default $p$ has the truth value of $q$, and
$np$ has the truth value of~$nq$:
\[
\ba l
   F_2= ((q \fand_m \neg_{\!s} \neg_{\!s} p) \rar_r p) 
      \fand_m ((nq \fand_m \neg_{\!s} \neg_{\!s}  np) \rar_r np). 
\ea
\]

%The formula represents that by default $p$ takes the same value as $q$.

Let ${\bf p} = \left\{p, np\right\}$ and ${\bf u}=\{u, nu\}$.
$F^*({\bf u})$ is
\[
\ba {rl}
    & \neg_{\!s}(p \fand_l np) \fand_m
                \neg_{\!s} \neg_{\!s}(p \for_l np)  \\
    &\fand_m ((q \fand_m \neg_{\!s} \neg_{\!s} p) \rar_r u) 
         \fand_m ((q\fand_m \neg_{\!s} \neg_{\!s} p) \rar_r p) \\ 
    & \fand_m ((nq \fand_m \neg_{\!s} \neg_{\!s}  np) \rar_r nu) 
      \fand_m ((nq \fand_m \neg_{\!s} \neg_{\!s}  np) \rar_r np). 
\ea 
\]
One can check that interpretation 
$I_1 = \left\{(p, x), (np, 1-x), (q, x), (nq, 1-x)\right\}$
($x$ is any value in $[0,1]$)
is a $1$-stable model of $F$ relative to $(p, np)$; 
interpretation $I_2 = \left\{(p, y), (np, 1-y), (q, x),
  (nq, 1-x)\right\}$, where $y \ne x$, is not. 

On the other hand, if we conjoin $F$ with $(\o{y}\rar_r p)\fand_m (\o{1-y}\rar_r np)$, the default behavior is overridden: $I_1$ is not a $1$-stable model of \hbox{$F\fand_m (\o{y} \rar_r p)\fand_m (\o{1-y}\rar_r np)$} relative to $(p,np)$, but $I_2$ is. 

This behavior is useful in expressing the commonsense law of inertia
involving fuzzy values. 
Suppose $q$ represents some fluent at time $t$, and $p$
represents the fluent at time $t\!+\!1$. Then $F$ states that, ``by
default, the fluent retains the previous value.'' The default value is
overridden if there is an action that sets $p$ to a different value. 
\end{example}

%---------------------------------------------------------------------
%\section{Further Examples} \label{sec:further}  \optional{sec:further}
%---------------------------------------------------------------------


%Recall the trust example in the introduction. 
% Suppose there are $n$ people. Everyone trusts each other to some
% degree. Normally, if a person $A$ trusts another person $B$, then
% $A$ tends to trust those who $B$ trusts, denoted by $C$, to a degree
% which is positively correlated to the degree to which $A$ trusts $B$
% and the degree to which $B$ trusts $C$. If nothing happens, the
% trust degrees do not change over time. But if a conflict between $A$
% and $B$ happens, there will be less trust between $A$ and $B$,
% depending on how severe the conflict is.

\begin{example}\label{ex:trust}
The trust example in the introduction can be formalized in the fuzzy stable model semantics as follows. 
Below $x$, $y$, $z$ are schematic variables ranging over people, and $t$ is
a schematic variable ranging over time steps. $\i{Trust}(x,y,t)$
is a fuzzy atom representing that ``$x$ trusts $y$ at time $t$.''
Similarly, $\i{Distrust}(x,y,t)$ is a fuzzy atom representing that 
``$x$ distrusts $y$ at time $t$.'' 
%Note that $\sneg$\ is just a part of the name of an atom, and not a
%connective. 

%Let $F_{TW} = F_1 \fand_m F_2 \fand_m \dots \fand_m F_{10}$ defined as
%follows.

The trust relation is reflexive: 
\[ 
  F_1 = \i{Trust}(x,x,t) .
\]
% ($t\in\{0,\dots,\i{maxstep}\}$):
%   F_1 = (\i{Trust}(x,y,t) \rar_r \i{Trust}(y,x,t))\fand_m, \\



% Distrust relation is symmetric  ($t\in\{0,\dots,\i{maxstep}$):
% \[
%    F_3 = (\i{Distrust}(x,y,t) \rar_r \i{Distrust}(y,x,t) 
% \]


The trust and distrust degrees are complementary, i.e., their sum is
$1$ (similar to Example~\ref{ex:default}): % ($t\in\{0,\dots,\i{maxstep}\}$): 
\[ 
\ba l 
   F_2 = \neg_{\!s} (\i{Trust}(x,y,t)\fand_l \i{Distrust}(x,y,t)),  \\
   F_3 = \neg_{\!s} \neg_{\!s} (\i{Trust}(x,y,t)\for_l\i{Distrust}(x,y,t)).

\ea 
\]

Initially, if $x$ trusts $y$ to degree $d_1$ and $y$ trusts $z$ to
degree $d_2$, then $x$ trusts $z$ to degree $d_1\times d_2$; further
the initial distrust degree is $1$ minus the initial trust degree.
\[ 
\ba l
   F_4 = \i{Trust}(x,y,0)\fand_p\i{Trust}(y,z,0)\rar_r\i{Trust}(x,z,0), \\
   F_5 = \neg_{\!s}\i{Trust}(x,y,0)\rar_r\i{Distrust}(x,y,0).
\ea 
\]

The inertia assumption (similar to Example~\ref{ex:default}): %  ($t\in\{0,\dots,\i{maxstep}\!-\!1\}$): 
\[
\ba l
    F_6 = \i{Trust}(x,y,t)\fand_m \neg_{\!s} \neg_{\!s}\i{Trust}(x,y,t\!+\!1)
          \rar_r \i{Trust}(x,y,t\!+\!1),  \\
    F_7 = \i{Distrust}(x,y,t)\fand_m \neg_{\!s} \neg_{\!s} \i{Distrust}(x,
    y, t\!+\!1) \rar_r \i{Distrust}(x,y,t\!+\!1).
\ea 
\]

A conflict increases the distrust
degree by the conflict degree: % ($t\in\{0,\dots,\i{maxstep}-1\}$):
\[
\ba l
   F_8= \i{Conflict}(x,y,t)\for_l \i{Distrust}(x,y,t) 
        \rar_r\i{Distrust}(x, y, t\!+\!1), \\
   F_9 = \neg_{\!s}(\i{Conflict}(x, y, t)\for_l \i{Distrust}(x, y, t)) 
         \rar_r \i{Trust}(x,y,t\!+\!1).
\ea 
\]

Let $F_{TW}$ be $F_1 \otimes_m F_2 \otimes_m \dots \otimes_m F_9$. 
%
%In the above formulas, we assume the variable $X$, $Y$ and $Z$ range
%over the $n$ people, the variable $T \in \left\{0, 1, \dots,
%  maxstep\right\}$ and the variable $ST \in \left(0, 1, \dots,
%  maxstep-1\right)$. 
%
%By a formula of the form $G(X)$, where $G$ is a
%formula involving variables $X$, we denote the formula $G(x_1) \fand
%\dots \fand G(x_n)$, where $x_1, \dots, x_n$ are elements in the
%domain of $X$.
%
%
Suppose we have the formula $F_{Fact}=\i{Fact}_1\fand_m\i{Fact}_2$ that
gives the initial trust degree.
\[
\ba l
   \i{Fact}_1 = \overline{0.8}\rar_r\i{Trust}(\i{Alice}, \i{Bob}, 0), \\
   \i{Fact}_2 = \overline{0.7}\rar_r\i{Trust}(\i{Bob}, \i{Carol}, 0).
\ea 
\]
Although there is no fact about how much $\i{Alice}$ trusts
$\i{Carol}$, any $1$-stable model of $F_{TW}\fand_m F_{Fact}$ assigns
value $0.56$ to the atom $\i{Trust}(\i{Alice}, \i{Carol}, 0)$. On the other
hand, the $1$-stable model assigns value $0$ to
$\i{Trust}(\i{Alice},\i{David},0)$ due to the closed world assumption under
the stable model semantics.

When we conjoin  $F_{TW}\fand F_{Fact}$ with 
$\overline{0.2}\rar\i{Conflict}(\i{Alice},\i{Carol}, 0)$, 
the $1$-stable model of 
$F_{TW}\fand_m
F_{Fact}\fand_m (\overline{0.2}\rar\i{Conflict}(\i{Alice},\i{Carol},
0))$
manifests that the trust degree between $\i{Alice}$ and $\i{Carol}$
decreases to $0.36$ at time $1$. More generally, if we have more actions that
change the trust degree in various ways, by specifying the entire
history of actions, we can determine the evolution of the trust
distribution among all the participants. Useful decisions can
be made based on this information. For example, $\i{Alice}$ may decide not to
share her personal pictures to those whom she trusts less than degree
$0.48$.
\end{example}

Note that this example, like Example~\ref{ex:default}, uses nested
connectives, such as $\neg_{\!s}\neg_{\!s}$, that are not available in
previous fuzzy ASP semantics, such as
\cite{lukasiewicz06fuzzy,janssen12reducing}.


\vspace{-0.3cm}


%---------------------------------------------------------------------
\section{Relation to Boolean-Valued Stable Models} \label{sec:rel-bool} \optional{sec:rel-bool}
%---------------------------------------------------------------------

The Boolean stable model semantics in Section~\ref{ssec:review-sm} can
be embedded into the fuzzy stable model semantics as follows: 

For any classical propositional formula $F$, define $F^\mi{fuzzy}$ to
be the fuzzy propositional formula obtained from $F$ by replacing $\bot$ with
$\o{0}$,\  $\top$ with $\o{1}$,\  $\neg$ with $\neg_{\!s}$,\ 
$\land$ with $\fand_m$,\ $\lor$ with $\for_m$, and 
$\rar$ with $\rar_s$.   We identify the signature of $F^\mi{fuzzy}$ with
the signature of $F$. Also, for any interpretation $I$, we
define the corresponding fuzzy interpretation $I^\mi{fuzzy}$ as 
\bi
\item  ${I^\mi{fuzzy}}(p) = 1$ if $I(p)=\true$; 
\item  ${I^\mi{fuzzy}}(p) = 0$ otherwise.
\ei

The following theorem tells us that the Boolean-valued stable model
semantics can be viewed as a special case of the fuzzy stable model semantics. 

\begin{thm}\label{thm:cl-fuzzy-sm}\optional{thm:cl-fuzzy-sm} 
% Consider the following choice of fuzzy operators:
% \begin{itemize}
% \item  $\fand(x, y) = min\left(x, y\right)$;
% \item  $\for(x, y) = max\left(x, y\right)$;
% \item  $\neg(x) = 1-x$;
% \item  $\rar(x, y) = max\left(1-x, y\right)$.
% \end{itemize}
% %
For any classical propositional formula $F$ and any classical
propositional interpretation $I$, $I$ is a stable model of $F$
relative to ${\bf p}$
iff $I^\mi{fuzzy}$ is a $1$-stable model of $F^\mi{fuzzy}$ relative to
${\bf p}$.
\end{thm}

\begin{example}
Let $F$ be the classical propositional formula $\neg p\rar q$. $F$ has
only one stable model
$I = \{q\}$.
% The fuzzy version of $F$, $F^\mi{fuzzy}=\neg_{\!s} p\rar_s q$. 
Clearly $I^\mi{fuzzy}=\{(p, 0), (q, 1)\}$ is a $1$-stable
model of $F^\mi{fuzzy}=\neg_{\!s} p\rar_s q$. 
%Furthermore, we can check that $F^\mi{fuzzy}$ has no other $1$-stable model. 
%$F^\mi{fuzzy}$ has
%only three $1$-models: $I^\mi{fuzzy}$, $J = \{(p, 1), (q,
%  1)\}$, and $K = \{(p, 1), (q, 0)\}$. $J$ and $K$ are
%not $1$-stable model of $F^\mi{fuzzy}$, as witnessed by
%$J^\prime=\{(p, 1), (q, 0)\}$ and $K^\prime=\{(p, 0), (q,
%  0)\}$, respectively.
\end{example}

Theorem \ref{thm:cl-fuzzy-sm} does not hold for an arbitrary choice of
operators, as illustrated by the following example.

\begin{example}
Let $F$ be the classical propositional formula $p\lor p$. 
Classical interpretation $I = \{p\}$ is a stable model of
$F$. However, $I^\mi{fuzzy}= \{(p,1)\}$ is not a stable model of 
$F' = p\for_l p$ because there is $J = \{(p, 0.5)\}$ such
that $I\cup J^p_q\models_1 q\for_l q$.
\end{example}

% \NB{
% We define the mapping $\i{defuz}(\cdot)$ from a fuzzy interpretation
% to a classical interpretation as follows: 
% \bi
% \item  $p^\mi{defuz(I)} = \true$ if $p^I = 1$; 
% \item  $p^\mi{defuz(I)} = \false$ otherwise. 
% \ei
% }

% Also for an interpretation $I=\left(p_1, \dots, p_n\right)$ of $F$,
% define the corresponding fuzzy interpretation as $I^\mi{fuzzy} = \left((p_1,
%   1), \dots, (p_n, 1), (q_1, 0), \dots, (q_m, 0)\right)$, where $q_1,
% \dots, q_m$ are atoms that appear in $F$ but do not appear in $I$.

\BOCC
\cmag
\begin{lemma}\label{lem_crisp_assignment}
For any fuzzy interpretation $I$ where all atoms are assigned only $0$
or $1$ and any $F$ where no numerical constant other than $\o{0}$
and $\o{1}$ can appear, $F^I \in \left(0, 1\right)$.
\end{lemma}


\begin{lemma}\label{lem_true_eq_1}
For a classical interpretation $I$ and a classical formula $F$, $I \models F$ if and only if $I^\mi{fuzzy} \models_1 F^\mi{fuzzy}$.
\end{lemma}
\EOCC

However, one direction of Theorem~\ref{thm:cl-fuzzy-sm} holds for
arbitrary choice of fuzzy operators. 
%5The following theorem assumes any
%choice of fuzzy conjunctions, disjunctions, negations and implications. 

\begin{thm} \label{thm:fuzzy-cl-sm}\optional{thm:fuzzy-cl-sm}
For any classical propositional formula $F$, let $F_1^\mi{fuzzy}$
be the fuzzy formula obtained from $F$ by replacing $\bot$ with
$\o{0}$,  $\top$ with $\o{1}$, $\neg$ with any fuzzy
negation symbol, $\land$ with any fuzzy conjunction symbol, $\lor$ with
any fuzzy disjunction symbol, and $\rar$ with any fuzzy
implication symbol. For any classical propositional interpretation
$I$, if $I^\mi{fuzzy}$ is a $1$-stable
model of $F_1^\mi{fuzzy}$ relative to ${\bf p}$, then $I$ is a stable
model of $F$ relative to ${\bf p}$.
\end{thm}


%\vspace{-5mm}
\vspace{-0.3cm}


%---------------------------------------------------------------------
\section{Relation to Other Approaches to  Fuzzy  ASP} \label{sec:rel-fuzzyasp}  \optional{sec:rel-fuzzyasp}
%---------------------------------------------------------------------

%\vspace{-3mm}
%---------------------------------------------------------------------
\subsection{Relation to Stable Models of Normal FASP Programs} \label{ssec:normal-fasp}\optional{ssec:normal-fasp}
%---------------------------------------------------------------------

A normal FASP program is a finite set of
rules of the form
\[ 
  a\ \ar\ b_1 \fand \dots \fand b_m \fand \neg b_{m+1}
\fand \dots \fand \neg b_n, 
\] 
where $n\geq m \geq 0$, $a, b_1, \dots, b_n$ are fuzzy atoms or
numeric constants in $[0,1]$, and $\fand$ is any fuzzy conjunction. We identify the rule with the fuzzy implication 
\[ 
   b_1\fand\dots\fand b_m\fand\neg_{\!s} b_{m+1}\fand\dots\fand
   \neg_{\!s} b_n\ \rar_r\ a. 
\] 
%where $\fand$ is any fuzzy conjunction.
%, and $\rar$ is the residual implicator of $\fand$. 

We say that a fuzzy interpretation $I$ of signature~$\sigma$ {\em
  satisfies} a rule $R$ if $R^I=1$. $I$ {\em satisfies} an FASP
program $\Pi$ if $I$ satisfies every rule in $\Pi$. 

%
According to \cite{lukasiewicz06fuzzy}, an interpretation $I$ is a
{\em fuzzy answer set} of a normal FASP program $\Pi$ if $I$ satisfies $\Pi$, and no interpretation $J$ such that $J<^\sigma I$ satisfies the reduct of
$\Pi$ w.r.t. $I$, which is the program obtained from $\Pi$ by replacing each
negative literal $\neg b$ with the constant for~$1-b^I$. 


% $I$ is a minimal
% model of a program $\Pi$ if $I\models\Pi$ and there is no interpretation
% $J < I$ such that $J \models \Pi$. 
% According to \cite{lukasiewicz06fuzzy}, an interpretation $M$ is an
% answer set of a normal FASP program $\Pi$ if $M$ is a minimal model of
% the reduct $\Pi^M$, which is obtained from $\Pi$ by replacing each
% negative literal $\neg b$ with the constant for $1-b^M$. 



\begin{thm}\label{thm:normal-fuzzysm}
For any normal FASP program $\Pi=\{r_1, \dots, r_n\}$, let $F$ be the
fuzzy formula $r_1\fand_m\dots\fand_m r_n$.
%, where $\fand$ is any fuzzy conjunction. 
An interpretation $I$ is a fuzzy answer set of $\Pi$ in
the sense of \cite{lukasiewicz06fuzzy} if and only if $I$ is a
$1$-stable model of $F$. 
\end{thm}

\NB{Yi: Example} 

\begin{example}
Let $\Pi$ be the following program
\[  
%\ba l
   p \leftarrow \neg q,\ \ \ \ 
   q \leftarrow \neg p.  
%\ea
\]
The answer sets of $\Pi$ according to~\cite{lukasiewicz06fuzzy} are 
$\{(p, x), (q, 1-x)\}$, where $x$ is any value in $\left[0, 1\right]$: 
the corresponding fuzzy formula $F$ is 
$(\neg_{\!s} q \rar_r p) \fand_m (\neg_{\!s} p \rar_r q)$;\ \ 
$F^*(u,v)$ is
%where $\fand$ can be any fuzzy conjunction.
\[
  F\fand_m ((\neg_{\!s} q \rar_r u)\fand_m (\neg_{\!s} p \rar_r v)). 
\]
One can check that the $1$-stable models of $F$ are also 
$\{(p, x), (q, 1-x)\}$, where $x \in \left[0, 1\right]$.
\end{example}

%\BOCCC
%---------------------------------------------------------------------
\subsection{Relation to Fuzzy Equilibrium Logic} \label{ssec:fuzzy-equil} \optional{ssec:fuzzy-equil}
%---------------------------------------------------------------------

%[[ Syntax is the same except it allows strong negation ]]
Like the fuzzy stable model semantics introduced in this paper, fuzzy
equilibrium logic~\cite{schockaert12fuzzy} generalizes fuzzy ASP programs to
arbitrary propositional formulas, but its definition is quite 
complex as it is based on a pair of intervals and considers strong
negation as one of the primary connectives. Nonetheless we show that fuzzy equilibrium logic is
essentially equivalent to the fuzzy stable model semantics where the
threshold is restricted to $1$ and all atoms are subject to minimization.
%\footnote{Strong negation can be simulated in our semantics using new atoms as illustrated in Example~\ref{ex:default}.}

\vspace{-5mm}
\subsubsection{Review: Fuzzy Equilibrium Logic}

We first review the definition of fuzzy equilibrium
logic from~\cite{schockaert12fuzzy}. The syntax is the same as the one
we reviewed in Section~\ref{ssec:review-fuzzy} except that a new
connective $\sneg\ $ (strong negation) may appear in front of atoms.\footnote{The definition from~\cite{schockaert12fuzzy} allows
  strong negation in front of any formulas. We restrict its occurrence
  only in front of atoms as usual in answer set programs.} 
%
For any fuzzy propositional signature~$\sigma$, a (fuzzy N5) \emph{valuation}
is a mapping from $\{h, t\}\times\sigma$ to subintervals of $\left[0,
  1\right]$ such that  $V(t,a)\subseteq V(h,a)$ for each atom $a\in\sigma$.
%
For $V(w, a)=\left[u, v\right]$, where $w\in\{h, t\}$, 
we write $V^-(w,a)$ to denote the lower bound~$u$ and 
$V^+(w,a)$ to denote the upper bound~$v$. The {\em truth value} of
a fuzzy formula under $V$ is defined as follows.
%
\begin{itemize}
\item  $V(w, \overline{c}) = [c,\ c]$ for any numeric constant $\overline{c}$;
\item  $V(w, \sneg a) = \left[1-V^+(w, a), 1-V^-(w, a)\right]$, where
  $\sim$ is the symbol for strong negation; 
\item  $V(w, F\fand G) = 
          [V^-(w,F)\fand V^-(w,G),\ \ V^+(w,F)\fand V^+(w,G)]$; \footnote{%
For readability, we write the infix notation $(x\odot y)$ in place of $\odot (x,y)$.}

\item $V(w, F\for G) = 
          [V^-(w,F)\for V^-(w,G),\ \ V^+(w,F)\for V^+(w,G)]$;

\item  $V(h,\neg F) = [1-V^-(t, F),\ \ 1-V^-(h, F)]$;

\item  $V(t,\neg F) = [1-V^-(t, F),\ \ 1-V^-(t, F)]$;

\item  $V(h, F\rar G) = 
          [min(V^-(h,F)\rar V^-(h,G), V^-(t,F)\rar V^-(t,G)),  \\
~~\hspace{8cm} V^-(h,F)\rar V^+(h, G)]$;

\item $V(t, F\rar G) = [V^-(t,F)\rar V^-(t,G),\ \
           V^-(t,F)\rar V^+(t,G)]$.
\end{itemize}

A valuation $V$ is a (fuzzy N5) model of a formula $F$ if $V^-(h,
F)=1$, which implies $V^+(h, F)=V^-(t, F)=V^+(t, F)=1$. For two
valuations $V$ and $V'$, we say $V'\preceq V$ if $V'(t, a)=V
(t, a)$ and $V(h, a)\subseteq V'(h, a)$ for all atoms $a$.
We say $V'\prec V$ if $V'\preceq V$ and $V'\ne V$.\ \ 
We say that a model $V$ of $F$ is {\em h-minimal} if there is no model $V'$ of $F$ such that $V'\prec V$.
%it is minimal w.r.t. $\preceq$ among all
%models of $F$. 
An h-minimal fuzzy N5 model $V$ of $F$ is a \emph{fuzzy
  equilibrium model} of $F$ if $V(h,a)=V(t,a)$ for all atoms $a$.

\BOCCC
For two fuzzy interpretations $I$, $J$ of signature $\sigma$
such that $J\leq^\sigma I$, define the N5 fuzzy valuation $V_{J, I}$
as $V_{J, I}(h, a) =
\left[a^J, 1\right]$, $V_{J, I}(t, a) = \left[a^I, 1\right]$ for all
atoms $a$ in $\sigma$. 
Since  $J\leq^\sigma I$, we have $a^J \leq a^I$, 
and $V_{J,I}(t,a)\subseteq V_{J,I}(h,a)$ for all atoms $a$. 

As in \cite{schockaert12fuzzy}, we assume that the fuzzy negation
$\neg$ is $\neg_{\!s}$. 
%
% \begin{lemma}\label{lem_model_tw}
% $F^{I\cup J^*} = V_{J, I}^-(t, F)$.
% \end{lemma}
%

% \begin{corollary}\label{model_eq}
% $I \models_1 F$ if and only if $V_{I, I}$ is a model of $F$.
% \end{corollary}
% \begin{lemma}
% $F^{* I\cup J^*} = V_{J, I}^-(h, F)$.
% \end{lemma}

The following proposition relates the notions used in the fuzzy
equilibrium models and the fuzzy stable models. 
\begin{prop}\label{prop:lem} 
\bi
\item[(a)]  $I\models_1 F$ if and only if $V_{I,I}$ is a model of $F$.
\item[(b)]  For ${\bf p}=\sigma$, $I\cup J^{\bf p}_{\bf q}\models_1 F^*({\bf q})$ 
            if and only if $V_{J, I}$ is a model of $F$.
\item[(c)]  For two interpretations $I$ and $J$, we have 
  $V_{J, I}\prec V_{I, I}$ if and only if $J < I$.
\ei
\end{prop}
\EOCCC

\BOCC
\begin{lemma}\label{clry_star_valuation}
$I\cup J^{\bf p}_{\bf q}\models_1 F^*({\bf q})$ 
if and only if $V_{J, I}$ is a model of $F$.
\end{lemma}

\begin{lemma}\label{lessthan_eq}
For two interpretations $I, J$ of the same formula $F$, $V_{J, I}
\prec V_{I, I}$ if and only if $J < I$.
\end{lemma}
\EOCC

\vspace{-3mm}
\subsubsection{In the Absence of Strong Negation}

%[[ Valuation as internavls look complicated ]]

We first establish the correspondence between fuzzy stable models and fuzzy
equilibrium models in the absence of strong negation. 
%The theorem below shows that the fuzzy stable model semantics and
%fuzzy equilibrium logic can be captured in each other can be
%reduced to fuzzy equilibrium logic semantics.
As in \cite{schockaert12fuzzy}, we assume that the fuzzy negation
$\neg$ is $\neg_{\!s}$. 

%We define mappings between fuzzy interpretations and fuzzy N5
%valuations as follows.

%\begin{itemize}
%\item 
For any valuation $V$, we define a fuzzy interpretation $I_V$ as 
$p^{I_V} = V^-(h,p)$ for each atom $p\in\sigma$.

%\item  For any interpretation $I$ of $\sigma$, we define
%  valuation $V_I$ as $V_I(w, a) = [a^I, 1]$ ($w\in\{h,t\}$). 
%\end{itemize}

\begin{thm}\label{thm:equil_sm_nostrneg}\optional{thm:equil_sm_nostrneg}
Let $F$ be a fuzzy propositional formula of $\sigma$ that contains no
strong negation.
\begin{itemize}
\item[(a)]  A valuation $V$ of~$\sigma$ is a fuzzy equilibrium model
  of $F$ iff $V^-(h,p)=V^-(t,p)$, $V^+(h,p)=V^+(t,p)=1$ for all
  atoms $p$ in $\sigma$ and $I_V$ is a $1$-stable model of $F$ relative to
  $\sigma$. 

\item[(b)] An interpretation $I$ of~$\sigma$ is a $1$-stable model of
  $F$ relative to $\sigma$ iff $I = I_V$ for some fuzzy equilibrium
  model $V$ of $F$. 
\end{itemize}
\end{thm}


%\EOCCC
\vspace{-3mm}
\subsubsection{In the Presence of Strong Negation}
%\subsection{Representing Strong Negation}

In this section we extend the relationship between fuzzy equilibrium
logic and our stable model semantics by allowing strong negation. 
This is done by simulating strong negation by new atoms in our
semantics. 
%at strong negation can be simulated by new atoms
%negation as an operator, strong negation of atoms can be handled by
%introducing new atoms and extra constraints. 
%We also show the
%equivalence between the representation of strong negation in our
%framework and its counterpart in fuzzy equilibrium logic, so that the
%correctness of this approach is verified.

%Strong negation in fuzzy equilibrium logic is defined as:
%\[
%  V(w, \sneg F)=\left[1-V^+(w, F),\ 1-V^-(w, F)\right].
%\]

Let $\sigma$ denote the signature. For a fuzzy formula $F$ over $\sigma$
that may contain strong negation, define $F^\prime$ over
$\sigma \cup \left\{np \mid p \in \sigma\right\}$ as the formula
obtained from $F$ by replacing all strong negations of atom $\sneg p$
with a new atom $np$. The transformation $nneg(F)$ (``{\sl n}o strong {\sl neg}ation'') is defined as
$nneg(F)=F^\prime \otimes_m \underset{p\in \sigma}{\bigotimes_m}\neg_s(p
\otimes_l np)$. 

\BOCC
\cgre
For a valuation $V$ of $\sigma$, define the valuation $nneg(V)$ of
$\sigma \cup \left\{np \mid p \in \sigma\right\}$ as $nneg(V)(w, p) =
\left[V^-(w, p), 1\right]$ and $nneg(V)(w, np) = \left[1-V^+(w, p),
  1\right]$ for all atoms $p \in \sigma$. Clearly for every valuation
$V$ of $\sigma$, there exists a corresponding interpretation
$I_{nneg(V)}$ of $\sigma \cup \left\{np \mid p \in \sigma\right\}$. On
the other hand, there exists interpretation $I$ of $\sigma \cup
\left\{np \mid p \in \sigma\right\}$ for which there is no
corresponding valuation $V$ of $\sigma$ such that $I=I_{nneg(V)}$.

\begin{example}
Suppose $\sigma=\left\{p\right\}$. The interpretation $I$ of $\sigma
\cup \left\{np \mid p \in \sigma\right\}$, defined by $I=\left\{(p,
  0.6), (np, 0.5)\right\}$, has no corresponding valuation $V$ of
$\sigma$ such that $I=I_{nneg(V)}$. Suppose, to the contrary, such
valuation exists. According to the definition, $V$ satisfies $V(h,
p)=\left[0.6, 0.5\right]$, which is impossible (since $\left[0.6,
  0.5\right]$ is not a valid interval).
\end{example}
\cbla 
\EOCC


%We extend the mappings between fuzzy N5 valuations of $F$ and 
%fuzzy interpretations for $\i{nneg}(F)$ as follows. 
%\begin{itemize}
%\item 
For any valuation $V$ of $\sigma$, we define the interpretation $I_V$
of $\sigma\cup\{np \mid p\in\sigma\}$ as 
\[
\begin{cases}
  p^{I_V} = V^-(h,p)  & \text{ for each $p\in\sigma$ };  \\
  np^{I_V} = 1 - V^+(h,p) & \text{ for each $np\notin\sigma$ }.
\end{cases}
\]

%\cgre
%\item  For any interpretation $I$ of $\sigma\cup\{np \mid p\in\sigma\}$,
%   we define valuation $V_I$ as $V_I(w,p)=[p^I, 1-np^I]$ for each
%   $p\in\sigma$.
%\cbla
%$a^{I_V} = V^-(h,a)$. 
%\end{itemize}

%[[ If $F$ contains no strong negation, then every stable model of $F$ has
%$np^I = 0$ for all $np\not\in\sigma$, so that these notions reduces to
%the notions in the previous section.  ]]

%If a formula contains  no strong negation
% Given an interpretation $I$ of $\sigma\cup\{np \mid p\in\sigma\}$ 
% %that satisfies [[... ]] for all $p\in\sigma$, 
% we define valuation $V_I$ as $V_I(w,p)=[p^I, 1-np^I]$.

% Given a valuation $V$, we define $I_V$ as 
% \[
% \begin{cases}
%   p^{I_V} = V^-(h,p)  & \text{ for each $p\in\sigma$ };  \\
%   np^{I_V} = 1 - V^+(h,p) & \text{ for each $np\notin\sigma$ }
% \end{cases}
% \]

\begin{thm}\label{thm:equil-sm}\optional{thm:equil-sm}
For any fuzzy formula $F$ of signature $\sigma$ that may contain strong negation, 
\begin{itemize}
\item[(a)]  A valuation $V$ of $\sigma$ is a fuzzy equilibrium model
  of $F$ iff $V(h,p)=V(t,p)$ for all atoms $p$ in~$\sigma$ and $I_V$ is a
  $1$-stable model of $\i{nneg}(F)$ relative to $\sigma\cup\{np \mid
  p\in\sigma\}$.

\item[(b)] An interpretation $I$ of $\sigma\cup\{np \mid p\in\sigma\}$ is
  a $1$-stable model of $\i{nneg}(F)$ relative to $\sigma\cup\{np \mid
  p\in\sigma\}$ iff $I=I_V$ for some fuzzy equilibrium model $V$ of
  $F$. 
\end{itemize}
\end{thm}


\BOCC
\begin{thm}\label{thm:equil-sm} 
For any fuzzy formula $F$ of signature $\sigma$ that may contain strong negation, 
\begin{itemize}
\item[(a)]  If $V$ is a fuzzy equilibrium model of $F$, then there is
  some interpretation $I$ such that $V=V_I$ and $I$ is a $1$-stable
  model of $\i{nneg}(F)$ relative to  $\sigma\cup\{np \mid
  p\in\sigma\}$.

\item[(b)]  If $I$ is a $1$-stable model of $\i{nneg}(F)$ relative to
  $\sigma\cup\{np \mid p\in\sigma\}$, then $V_I$ exists and is a fuzzy equilibrium model of $F$. 
\end{itemize}
\end{thm}


\begin{thm}\label{thm:equil-sm} 
For any fuzzy formula $F$ of signature $\sigma$ that may contain strong negation, 
\begin{itemize}
\item[(a)]  If $V$ is a fuzzy equilibrium model of $F$, then $I_V$ is a $1$-stable model of $\i{nneg}(F)$ relative to  $\sigma\cup\{np \mid p\in\sigma\}$.

\item[(b)]  If $I$ is a $1$-stable model of $\i{nneg}(F)$ relative to
  $\sigma\cup\{np \mid p\in\sigma\}$, then $V_I$ is a fuzzy equilibrium model of $F$. 
\end{itemize}
\end{thm}

\cgre
\begin{thm}\label{thm:equil-sm}
Let $F$ be a fuzzy propositional formula. 
\begin{itemize}
\item[(a)]  If $V$ is an equilibrium model of $F$, then $I_{nneg(V)}$
  is a $1$-stable model of $nneg(F)$.
\item[(b)]  If $I$ is a $1$-stable model of $nneg(F)$, then there
  exists $V$ such that $I=I_{nneg(V)}$ and $V$ is an equilibrium model
  of $F$.
\end{itemize}
\end{thm}
\cbla
\EOCC


\BOCC
\begin{thm}\label{thm:equili2sm_sn}
If $V$ is an equilibrium model of $F$, then $I_{nneg(V)}$ is a $1$-stable model of $nneg(F)$.
\end{thm}

\begin{thm}\label{thm:sm2equili_sn}
If $I$ is a $1$-stable model of $nneg(F)$, then there exists $V$ such that $I=I_{nneg(V)}$ and $V$ is an equilibrium model of $F$.
\end{thm}
\EOCC

%From Theorem \ref{thm:equili2sm_sn} and Theorem
%\ref{thm:sm2equili_sn}, we know that 
%[[ Thus there is a 1-1 correspondence of the equilibrium models of a
%formula $F$ and the $1$-stable models of $nneg(F)$. The next example
%illustrates how to represent strong negation of atoms in fuzzy stable
%model semantics.]]

\begin{example}
For fuzzy formula 
$F=(\overline{0.2} \rar_r p)\fand_m (\overline{0.3} \rar_r np)$, 
formula $nneg(F)$ is
\[ (\overline{0.2} \rar_r p)\fand_m
   (\overline{0.3} \rar_r np) \fand_m
  \neg_s(p \fand_l np).
\]
One can check that the valuation $V$ defined as $V(w, p)=\left[0.2, 0.7\right]$ is the only equilibrium model of $F$, and the interpretation $I_V=\left\{(p, 0.2), (np, 0.3)\right\}$ is the only $1$-stable model of $nneg(F)$.
\end{example}

This idea of eliminating strong negation in favor
of new atoms was used in Examples~\ref{ex:default} and \ref{ex:trust}.

\vspace{-0.3cm}

%---------------------------------------------------------------------
\section{Properties of Fuzzy Stable Models} \label{sec:properties} 
   \optional{sec:properties}
%---------------------------------------------------------------------

In this section, we show that several well-known properties of the Boolean stable model
semantics can be naturally extended to the fuzzy stable model
semantics. 

\subsection{Alternative Definition of $F^*$}


\begin{prop}\label{lem_star_negation}
For any fuzzy formulas $F$, $G$ and any fuzzy interpretations $I$, $J$ such that $J\le^{\bf p} I$, 
\bi
 \item  $I\cup J^{\bf p}_{\bf q}\models_y \neg F^*({\bf q})\fand_m
   \neg F$\ ~~iff~~\ $I\cup J^{\bf p}_{\bf q}\models_y \neg F$;
 \item  $I\cup J^{\bf p}_{\bf q}\models_y (F^*\fand G^*)({\bf q})\fand_m (F\fand
   G)$\ ~~iff~~\ $I\cup J^{\bf p}_{\bf q}\models_y (F^*\fand G^*)({\bf q})$;
 \item  $I\cup J^{\bf p}_{\bf q}\models_y (F^*\for G^*)({\bf q})\fand_m (F\for
   G)$\ ~~iff~~\ $I\cup J^{\bf p}_{\bf q}\models_y (F^*\for G^*)({\bf q})$.
\ei
\end{prop}

This proposition tells us that $F^*$ in Section~\ref{sec:definition}
can be equivalently defined by treating the fuzzy operators in the
uniform way without affecting stable models.
%for the case of fuzzy operators as
%follows, where the operators are uniformly treated: 

\bi
\item  $(\neg F)^* = \neg F^*\fand_m \neg F$;
\item  $(F\odot G)^* = (F^*\odot G^*)\fand_m (F\odot G)$ for any
  binary operator $\odot$.
\ei




\subsection{Theorem on Constraints}

In answer set programming, constraints---rules with $\bot$ in the
head---play an important role in view of the fact that adding a
constraint eliminates the stable models that ``violate'' the
constraint. 
The following theorem is the counterpart of Theorem~3
from~\cite{ferraris11stable} for fuzzy propositional formulas. 

\begin{thm}\label{thm:constraint}\optional{thm:constraint}
For any fuzzy formulas $F$ and $G$, $I$ is a $1$-stable model of
\hbox{$F\fand\neg G$} (relative to ${\bf p}$) if and only if $I$ is a
$1$-stable model of $F$ (relative to ${\bf p}$) and $I\models_1\neg G$.
\end{thm}

\begin{example}
Consider $F=(\neg_{\!s} p \rar_r q)\fand_m(\neg_{\!s} q \rar_r p)\fand_m \neg_{\!s} p$. Formula~$F$ has only one $1$-stable model $I = \{(p, 0), (q, 1)\}$, which is the only $1$-stable model of \\
\hbox{$(\neg_{\!s} p \rar_r q)\fand_m(\neg_{\!s} q \rar_r p)$} that satisfies $\neg_{\!s} p$ to degree $1$.
\end{example}



If we consider a more general $y$-stable model, then only one
direction holds.

\begin{thm}\label{thm:constraint2} \optional{thm:constraint2}
For any fuzzy formulas $F$ and $G$, if $I$ is a $y$-stable model of 
$F\fand\neg G$ (relative to ${\bf p}$), then $I$ is a $y$-stable model
of $F$ (relative to ${\bf p}$) and $I\models_y\neg G$.
\end{thm}

\begin{example}
The other direction, that is, ``if $I$ is a $y$-stable model of $F$ and 
$I \models_y\neg G$, then $I$ is a $y$-stable model of $F\fand \neg
G$,'' does not hold in general. For example, consider $F=G=p$ and
$\fand$ to be $\fand_l$, and 
%consider $p\fand_l\neg_{\!s} p$ and 
interpretation $I=\{(p,0.4)\}$. 
Clearly $I$ is a $0.4$-stable model of $p$ and $I\models_{0.4} \neg
p$, but $I$ is not a $0.4$-stable model of $p\fand_l \neg p$. In fact, $I$ is not even a $0.4$-model of the formula.
\end{example}


\subsection{Theorem on Choice Formulas} 

In the Boolean stable model semantics, formulas of the form $p\lor\neg p$
are called {\em choice formulas}, and adding them to the program makes 
atoms $p$ exempt from minimization. Choice formulas have been shown to
be useful in composing a program in the ``Generate-and-Test'' method.
This section shows their counterpart in the fuzzy stable model semantics. 

For any fuzzy atom $p$, $\i{Choice}(p)$ stands for 
$p\for_l \neg_{\!s} p$.
%  where $\for_l(x, y) = {\rm min}(x+y, 1)$, and $\neg_{\!s}(x) = 1-x$.
For any list ${\bf p}=(p_1,\dots p_n)$ of fuzzy atoms,
$\i{Choice}({\bf p})$ stands for 
$\i{Choice}(p_1)\fand\dots\fand\i{Choice}(p_n)$,
where $\fand$ is any fuzzy conjunction. 

The following proposition tells that choice formulas are
tautological. 

\begin{prop}\label{lem_choice_tautology}
For any fuzzy interpretation $I$ and any list ${\bf p}$ of fuzzy
atoms, \hbox{$I\models_1\i{Choice}({\bf p})$}.
\end{prop}

% \begin{lemma}\label{choice_tautology}
% For any interpretation $I$ and any atom $a$, $I \models_1\i{Choice}(a)$.
% \end{lemma}
% \begin{proof}
% \begin{align}
% \nonumber choice(a)^I &= (a \for_l \neg_{\!s} a)^I \\
% \nonumber &= min\left(a^I + 1-a^I, 1\right) \\
% \nonumber &= 1
% \end{align}
% \end{proof}

Theorem~\ref{thm:choice} is an extension of Theorem~2
from~\cite{ferraris11stable}. % to the fuzzy stable model semantics. 

\begin{thm} \label{thm:choice}
\bi
\item[(a)] If $I$ is a $y$-stable model of $F$ relative to ${\bf
    p}\cup {\bf q}$, then $I$ is a $y$-stable model of $F$ relative to
  ${\bf p}$.
\item[(b)] $I$ is a $1$-stable model of $F$ relative to ${\bf p}$ iff 
   $I$ is a $1$-stable model of $F\fand\i{Choice}({\bf q})$ relative
   to ${\bf p}\cup {\bf q}$.
\ei
\end{thm}


Theorem~\ref{thm:choice}~(b) does not hold for arbitrary threshold
$y$ (i.e., if ``$1-$'' is replaced with ``$y-$''). For example, consider
$F=\neg_s\neg_s q$ and $I = \left\{(q, 0.5)\right\}$. Clearly $I$ is a
$0.5$-model of $F$, and thus $I$ is a $0.5$-stable model of $F$
relative to $\emptyset$. However, $I$ is not a $0.5$-stable model of
$F\fand_m {\i Choice(q)}=\neg_s \neg_s q\fand_m (q\for_l \neg_s q)$
relative to $\emptyset \cup \left\{q\right\}$, as witnessed by $J =
\left\{(q, 0)\right\}$.

Since the $1$-stable models of $F$ relative to $\emptyset$ are the models
of $F$, it follows from Theorem~\ref{thm:choice} (b)  that the {\em $1$-stable models} of $F\fand\i{Choice}(\sigma)$ relative to $\sigma$ are exactly the {\em $1$-models} of $F$.
\begin{cor}
Let $F$ be a fuzzy formula of a finite signature $\sigma$.
$I$ is a $1$-model of $F$ iff $I$ is a
$1$-stable model of~$F\fand\i{Choice}(\sigma)$ relative to~$\sigma$.
\end{cor}


\begin{example}
Consider the fuzzy formula $F=\neg_{\!s} p \rar_r q$. Although any
interpretation $I$ that satisfies $1-p^I \leq q^I$ is a $1$-model of
$F$, among them only $\{(p, 0), (q, 1)\}$ is a $1$-stable model of
$F$. However, we check that all $1$-models of $F$ are exactly the
$1$-stable models of $G = F\fand_m Choice(p) \fand_m Choice(q)$:\ \  
% \begin{align}
% \nonumber G =  F\fand_m Choice(p) \fand_m Choice(q) &= (\neg_{\!s} p \rar_r q) \fand_m (p\for_l \neg_{\!s} p) \fand_m (q\for_l \neg_{\!s} q)
% \end{align}
$G^*(u,v)$ is 
\[ 
   (\neg_{\!s} p \rar_r q) \fand_m (\neg_{\!s} p \rar_r v) \fand_m (u\for_l
   \neg_{\!s} p) \fand_m (v\for_l \neg_{\!s} q)
\]
and for $K = I\cup J^{pq}_{uv}$, 
\[
%   G^*(u,v)^K = \left(\o{1} \fand_m (\o{(1-p^K)}\rar_r\o{v^K}) \fand_m   
%           (\o{u^K}\for_l\o{(1-p^K)}) \fand_m (\o{v^K}\for_l %\o{(1-q^K)})\right)^K
   G^*(u,v)^K = 1 \fand_m ((1-p^K) \rar_r v^K) \fand_m (u^K\for_l
   (1-p^K)) \fand_m (v^K\for_l (1-q^K)).
\]
So, for $K$ to satisfy $G^*(u,v)$ to degree $1$, $u^K$ should be 
at least $p^K$ and $v^K$ should be at least $q^K$. So there does not
exist $J <^{pq} I$ such that $I\cup J^{pq}_{uv} \models_1 G^*(u,v)$,
from which it follows that $I$ is a $1$-stable model of $G$.
\end{example}

\vspace{-0.3cm}

%---------------------------------------------------------------------
\section{Other Related Work} \label{sec:related_work} \optional{sec:related_work}
%---------------------------------------------------------------------

% Equili_logic and semantics for normal programs
% Other semantics of FASP (syntax; reduct-based approaches; unfounded-set based approaches; weighted vs. unweighted)
Several approaches to incorporating fuzziness into the answer
set programming framework have been proposed. In this paper, we have
formally compared our approach to \cite{schockaert12fuzzy} and
\cite{lukasiewicz06fuzzy}.
%, and shown that our approach is equivalent
%to fuzzy equilibrium logic \cite{schockaert12fuzzy} when strong
%negation is applied to atoms only and can capture
%the reduct-based semantics of normal fuzzy answer set
%programs \cite{lukasiewicz06fuzzy}. There are a variety of other
%approaches. 
Most of them consider the specific syntax where each
formula is of the rule form $h\leftarrow B$ where $h$ is an atom and
$B$ is a formula
\cite{vojtas01fuzzy,damasio01monotonic,medina01multi-adjoint,damasio01antitonic}. 
Among them, \cite{vojtas01fuzzy,damasio01monotonic,medina01multi-adjoint} allow $B$ to be any arbitrary formula
corresponding to an increasing function whose arguments are the atoms
appearing in the formula. ~\cite{damasio01antitonic} allows $B$
to correspond to either an increasing function or a decreasing
function. ~\cite{madrid08towards} considers the normal program
syntax, i.e., each rule is of the form $l_0 \leftarrow l_1 \fand \dots
\fand l_m \fand not\ l_{m+1} \fand \dots \fand not\ l_{n}$, where each
$l_i$ is an atom or the strong negation of an atom. In terms of
semantics, most of the previous works rely on the notion of immediate
consequence operator and relate the fixpoint of this operator to the
minimal model of a positive program.\footnote{We call a program
  positive if it does not contain any default negation.} Similar to
the approach ~\cite{lukasiewicz06fuzzy} has adopted, the answer set of
a positive program is defined as its minimal model, while an answer
set of a non-positive program is defined in terms of the minimal model
of the reduct, which is a positive program obtained based on the
normal program and the specific interpretation being
checked. ~\cite{nieuwenborgh07anintroduction} has proposed a semantics based on
the notion of an unfounded set.
% rather than the reduct. But it turns out
%the unfounded-set based semantics coincides with the reduct-based
%definition in normal programs(~\cite{janssen12reducing}). 
%As shown in this paper, our semantics does not rely on the notion of
%reduct, but it captures the reduct-based definition when normal programs %are
%considered.

It is worth noting that some of the related works have discussed
the so-called residuated programs
\cite{vojtas01fuzzy,damasio01monotonic,medina01multi-adjoint,madrid08towards}, where each rule $h
\leftarrow B$ is assigned a weight $\theta$, and a rule is satisfied by an
interpretation $I$ if $I(h \leftarrow B) \geq \theta$. According to
\cite{damasio01monotonic}, this class of programs is able to
capture many other logic programming paradigms, such as possibilistic
logic programming, hybrid probabilistic logic programming, generalized
annotated logic programming. Furthermore, as shown in
~\cite{damasio01monotonic}, a weighted rule $(h \leftarrow B,
\theta)$ can be simulated by $h \leftarrow B \fand \theta$, where
$(\fand, \leftarrow)$ forms an ``adjoint pair.''


% Other approach dealing with strong negation
%In our framework, we do not have any operator corresponding to strong
%negation, but instead we introduce new atoms and extra formulas to
%represent strong negation. 

It is well known in the Boolean stable model semantics that strong
negation can be represented in terms of new atoms
\cite{ferraris11stable}. 
Our adaptation in the fuzzy stable model semantics is similar to the
method from~\cite{madrid08towards}, in which the consistency of an interpretation is guaranteed by imposing the extra restriction $I(\sneg p) \leq\ \sneg I(p)$ for all atom $p$. Strong negation and consistency have also been studied in \cite{madrid11measuring,madrid09oncoherence}.

% Other semantics dealing with fuzzyness (annotated ASP, possibilistic
% ASP?)
\BOCC
In addition to fuzzy answer set programming, there are other paradigms
developed to handle many-valued logic. For example,
\cite{straccia06annotated} has proposed a logic programming framework
where each literal is annotated by a real-valued interval, in this way
the non-monotonicity of answer set programming and many-valued logic
are combined. Another example is possibilistic
logic \cite{dubois04possibilistic}, where each propositional
symbol is associated with two real values in the interval $\left[0,
  1\right]$ called necessity degree and possibility degree. Although
these semantics handle fuzziness based on quite different ideas, it
has been shown that these paradigms can be captured by fuzzy answer
set programming~\cite{damasio01monotonic}.
\EOCC

\vspace{-0.3cm}


%---------------------------------------------------------------------
\section{Conclusion} \label{sec:conclusion} \optional{sec:conclusion}
%---------------------------------------------------------------------

%[[ exapnd conclusion ]] 

We introduced a stable model semantics for fuzzy propositional
formulas, which generalizes both the Boolean stable model semantics
and fuzzy propositional logic. The syntax is the same as the syntax of
fuzzy propositional logic, but the semantics defines {\em stable
  models} instead of {\em models}. The formalism allows highly
configurable default reasoning involving fuzzy truth values. 
Our semantics, when we restrict threshold to be~$1$ and assume all
atoms to be subject to minimization, is essentially equivalent to
fuzzy equilibrium logic, but is much simpler. To the best of our
knowledge, our representation of the commonsense law of inertia involving
fuzzy values is new. The representation uses nested fuzzy operators,
which are not available in other fuzzy ASP semantics for a restricted syntax.

We showed that several traditional results in answer set programming can be naturally extended to this formalism, and expect that more results can be carried over. 
%Our semantics is equivalent to fuzzy equilibrium logic, but is much
%more simpler. 
Future work includes
implementing this language using mixed integer programming solvers or
bilevel programming solvers~\cite{alviano13fuzzy}. 

%The future work includes comparison with fuzzy equilibrium logic, and
%Janssens' work. 

\smallskip\noindent
{\bf Acknowledgements}\ \ 
We are grateful to Joseph Babb, Michael Bartholomew, Enrico Marchioni, and the anonymous
referees for their useful comments and discussions related to this paper. 
This work was partially supported by the National Science Foundation under Grant IIS-1319794 and by the South Korea IT R\&D program MKE/KIAT
2010-TD-300404-001.

%\bibliographystyle{named}
\bibliographystyle{splncs}
%\bibliography{bib,bib2}

\begin{thebibliography}{10}

\bibitem{lif08}
Lifschitz, V.:
\newblock What is answer set programming?
\newblock In: Proceedings of the AAAI Conference on Artificial Intelligence,
  MIT Press (2008)  1594--1597

\bibitem{lukasiewicz06fuzzy}
Lukasiewicz, T.:
\newblock Fuzzy description logic programs under the answer set semantics for
  the semantic web.
\newblock In Eiter, T., Franconi, E., Hodgson, R., Stephens, S., eds.: RuleML,
  IEEE Computer Society (2006)  89--96

\bibitem{janssen12reducing}
Janssen, J., Vermeir, D., Schockaert, S., Cock, M.D.:
\newblock Reducing fuzzy answer set programming to model finding in fuzzy
  logics.
\newblock TPLP \textbf{12}(6) (2012)  811--842

\bibitem{vojtas01fuzzy}
Vojt{\'a}s, P.:
\newblock Fuzzy logic programming.
\newblock Fuzzy Sets and Systems \textbf{124}(3) (2001)  361--370

\bibitem{damasio01monotonic}
Dam{\'a}sio, C.V., Pereira, L.M.:
\newblock Monotonic and residuated logic programs.
\newblock In Benferhat, S., Besnard, P., eds.: ECSQARU. Volume 2143 of Lecture
  Notes in Computer Science., Springer (2001)  748--759

\bibitem{medina01multi-adjoint}
Medina, J., Ojeda-Aciego, M., Vojt{\'a}s, P.:
\newblock Multi-adjoint logic programming with continuous semantics.
\newblock In: Proceedings of International Conference on Logic Programming and
  Nonmonotonic Reasoning ({LPNMR}). (2001)  351--364

\bibitem{damasio01antitonic}
Dam{\'a}sio, C.V., Pereira, L.M.:
\newblock Antitonic logic programs.
\newblock In: Proceedings of International Conference on Logic Programming and
  Nonmonotonic Reasoning ({LPNMR}). (2001)  379--392

\bibitem{nieuwenborgh07anintroduction}
Nieuwenborgh, D.V., Cock, M.D., Vermeir, D.:
\newblock An introduction to fuzzy answer set programming.
\newblock Ann. Math. Artif. Intell. \textbf{50}(3-4) (2007)  363--388

\bibitem{madrid08towards}
Madrid, N., Ojeda-Aciego, M.:
\newblock Towards a fuzzy answer set semantics for residuated logic programs.
\newblock In: Web Intelligence/IAT Workshops, IEEE (2008)  260--264

\bibitem{ferraris11stable}
Ferraris, P., Lee, J., Lifschitz, V.:
\newblock Stable models and circumscription.
\newblock Artificial Intelligence \textbf{175} (2011)  236--263

\bibitem{hajek98mathematics}
Hajek, P.:
\newblock Mathematics of Fuzzy Logic.
\newblock Kluwer (1998)

\bibitem{schockaert12fuzzy}
Schockaert, S., Janssen, J., Vermeir, D.:
\newblock Fuzzy equilibrium logic: Declarative problem solving in continuous
  domains.
\newblock ACM Trans. Comput. Log. \textbf{13}(4) (2012) ~33

\bibitem{madrid11measuring}
Madrid, N., Ojeda-Aciego, M.:
\newblock Measuring inconsistency in fuzzy answer set semantics.
\newblock IEEE T. Fuzzy Systems \textbf{19}(4) (2011)  605--622

\bibitem{madrid09oncoherence}
Madrid, N., Ojeda-Aciego, M.:
\newblock On coherence and consistence in fuzzy answer set semantics for
  residuated logic programs.
\newblock In Ges{\`u}, V.D., Pal, S.K., Petrosino, A., eds.: WILF. Volume 5571
  of Lecture Notes in Computer Science., Springer (2009)  60--67

\bibitem{alviano13fuzzy}
Alviano, M., Pe{\~n}aloza, R.:
\newblock Fuzzy answer sets approximations.
\newblock TPLP \textbf{13}(4-5) (2013)  753--767

\end{thebibliography}



%\newpage
%\appendix
%\include{fuzzysm-jelia-appendix-0714}

\end{document}
